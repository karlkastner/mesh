\documentclass[11pt,twoside,a4paper]{article}
%{book}

% This is an automatically generated file.
% Do not edit it.
% Changes to this file are not preserved!

\usepackage{tocloft}
\usepackage{hyperref}
\usepackage{listings}
\lstset{
basicstyle=\small\ttfamily,
columns=flexible,
breaklines=true
}
\setlength{\cftsubsecnumwidth}{3.5em}

\title{Manual for Package:
mesh\protect\\Revision 1:3M
}
\author{Karl K\"astner}
%\date{}

\begin{document}

\maketitle

\tableofcontents

% licence
% abstract


\section{@StructuredMesh}
\subsection{StructuredMesh}
${}$
\begin{lstlisting}
 structured mesh processing
 compatible with Delft3D
 also provides set-up of discretisation matrices

\end{lstlisting}
\subsection{apply\_boundary\_condition}
${}$
\begin{lstlisting}
 apply boundary condition and the four sides of the domain
 TODO: allow for interior boudaries

\end{lstlisting}
\subsection{bc\_from\_shp}
${}$
\begin{lstlisting}
 read boundary condition from shape file

\end{lstlisting}
\subsection{bc\_index}
${}$
\begin{lstlisting}
 TODO this is deprecated
 generate indices for boundary edges

\end{lstlisting}
\subsection{bc\_isinvalid}
${}$
\begin{lstlisting}
 check boundary conditions for stacked domains

\end{lstlisting}
\subsection{block}
${}$
\begin{lstlisting}
 stack multiple meshes to complex domain

\end{lstlisting}
\subsection{boundary\_chain}
${}$
\begin{lstlisting}
 return chain of boundary points

\end{lstlisting}
\subsection{boundary\_direction}
${}$
\begin{lstlisting}
 return direction of boundary segment

\end{lstlisting}
\subsection{boundary\_indices}
${}$
\begin{lstlisting}
 indices of boundary segments
 id : index of boundary point
 jd : index of 

\end{lstlisting}
\subsection{cat}
${}$
\begin{lstlisting}

\end{lstlisting}
\subsection{centreline}
${}$
\begin{lstlisting}
 domain (channel) centreline along chosen dimension

\end{lstlisting}
\subsection{child}
${}$
\begin{lstlisting}
 hierarchical mesh generation (for bifurcations)

\end{lstlisting}
\subsection{copy}
${}$
\begin{lstlisting}

\end{lstlisting}
\subsection{corner\_indices}
${}$
\begin{lstlisting}
 indices of domain corners

\end{lstlisting}
\subsection{cut\_from\_domain}
${}$
\begin{lstlisting}
 cut subdomain

\end{lstlisting}
\subsection{export\_delft3d\_bnd}
${}$
\begin{lstlisting}
 export the boundary in delft3d compatible format 

\end{lstlisting}
\subsection{export\_delft3d\_dep}
${}$
\begin{lstlisting}
 export bathymetry data in Delft3D dep-format

\end{lstlisting}
\subsection{export\_delft3d\_grd}
${}$
\begin{lstlisting}
 export mesh in deltares delft3D grd file format

\end{lstlisting}
\subsection{export\_delft3d\_ini}
${}$
\begin{lstlisting}
 export delft3D compatible initial condition file

\end{lstlisting}
\subsection{export\_shp}
${}$
\begin{lstlisting}
 export mesh elements as shape file

\end{lstlisting}
\subsection{extend\_straight\_reach}
${}$
\begin{lstlisting}

\end{lstlisting}
\subsection{extract\_elements}
${}$
\begin{lstlisting}
 element indices from grid

\end{lstlisting}
\subsection{flip\_dimension}
${}$
\begin{lstlisting}
 flip left and right or top and down

\end{lstlisting}
\subsection{from\_1d\_mesh}
${}$
\begin{lstlisting}
 convert a 1D mesh to 2D mesh consisting of quadrilaterals

\end{lstlisting}
\subsection{generate\_bifurcation}
${}$
\begin{lstlisting}
 creates a mesh for bifurcation with bluff, which is required for delft3d grids
 TODO do not fix indices
 TODO determine p individually
 bank : bankline shapefile
 nn : number of points across branches
 ds: spacing along s
 p : fraction of right side branch
 level : generate hierarchical mesh,
	  grid points in each branch will be 2^n+1,
	  and sub meshes until level 1 will be generated

 for lower levels the connecting volumes remain narrow,
 as the two volumes left and right of the division line are not scaled
 -> post smoothing required

 nn: n=6; for idx=1:5; n(end+1)  = 2*(n(end)-3)+3, end
 ns: n=18; for idx=1:5; n(end+1) = 2*(n(end)-2)+2, end (should be improved to 2*(n-1)+1

\end{lstlisting}
\subsection{generate\_disk}
${}$
\begin{lstlisting}
 generate semicircular domain

\end{lstlisting}
\subsection{generate\_from\_centreline}
${}$
\begin{lstlisting}
 generate a mesh from a given centreline
 TODO : avoid crossing of inner bed points in sharp bends

\end{lstlisting}
\subsection{generate\_rectangle}
${}$
\begin{lstlisting}
 discretize a rectangular domain

\end{lstlisting}
\subsection{generate\_structured\_grid}
${}$
\begin{lstlisting}
 generate a structured mesh consisting of several sub-meshes

\end{lstlisting}
\subsection{grid\_block}
${}$
\begin{lstlisting}
 mesh a subdomain

\end{lstlisting}
\subsection{improve}
${}$
\begin{lstlisting}
 improve (smooth) the mesh

\end{lstlisting}
\subsection{interp\_elem2point}
${}$
\begin{lstlisting}
 interpolate values sampled at element centres to element corners
 TODO allow also interpolation to u and v points

\end{lstlisting}
\subsection{mesh\_polygon}
${}$
\begin{lstlisting}
 mesh a 1D channel, where boundaries are given as polygon
 TODO, this should better use voronoi-tesselation (see centreline class)

\end{lstlisting}
\subsection{orthogonality}
${}$
\begin{lstlisting}
 orthogonality of elements

\end{lstlisting}
\subsection{orthogonalize}
${}$
\begin{lstlisting}
 orthogonalize mesh
 set x of point coordinates to 1/2

\end{lstlisting}
\subsection{plot}
${}$
\begin{lstlisting}
 plot the mesh

\end{lstlisting}
\subsection{plot\_boundary}
${}$
\begin{lstlisting}
 plot the mesh boundary

\end{lstlisting}
\subsection{plot\_coupling}
${}$
\begin{lstlisting}
 plot connected vertices, see vertex_connection_matrix.m

\end{lstlisting}
\subsection{plot\_orthogonality}
${}$
\begin{lstlisting}
 plot mesh with edges colored by orthogonality condition

\end{lstlisting}
\subsection{quiver}
${}$
\begin{lstlisting}
 quiver plot of velocity

\end{lstlisting}
\subsection{read\_delft3d\_dep}
${}$
\begin{lstlisting}
 depth in dat file is defined at volume centres (water leve point)
 first row, first column and last column are buffer
 but nast colum is not (only when outflow?)

\end{lstlisting}
\subsection{read\_delft3d\_grd}
${}$
\begin{lstlisting}
 read mesh in delft3D grd format

\end{lstlisting}
\subsection{smooth\_cubic}
${}$
\begin{lstlisting}
 cubically smooth the mesh coordinates

\end{lstlisting}
\subsection{smooth\_curvilinear}
${}$
\begin{lstlisting}
 smooth the mesh
relax = (10+relax)/11;
relax = min(0.5,relax);

\end{lstlisting}
\subsection{smooth\_laplacian}
${}$
\begin{lstlisting}
 smooth the mesh coordinates

 better than before, but causes dn in inner bends to be narrower than in outer bends
 (straightens the lines)
 better smooth p: i.e. fractional distance from left to right,
 this is complicated at the bif
 better: two neighbour smooth: smooth dn and ds with left/right, top bottom only 

\end{lstlisting}
\subsection{smooth\_simple}
${}$
\begin{lstlisting}

 smooth the mesh coordinates

\end{lstlisting}
\subsection{smooth\_sn}
${}$
\begin{lstlisting}

 smooth the mesh coordinates


\end{lstlisting}
\subsection{snap}
${}$
\begin{lstlisting}
 snap two meshes that connect at their domain boundaries

\end{lstlisting}
\subsection{statistic}
${}$
\begin{lstlisting}
 compute mesh statistics

\end{lstlisting}
\subsection{to\_unstructured\_mesh}
${}$
\begin{lstlisting}
 convert to unstructured mesh

\end{lstlisting}
\subsection{transpose\_dimension}
${}$
\begin{lstlisting}
 transpose dimensions

\end{lstlisting}
\subsection{vertex\_connection\_matrix}
${}$
\begin{lstlisting}
 connectivity of neighbouring vertices
 TODO same for elements

\end{lstlisting}
\section{@UnstructuredMesh}
\subsection{UnstructuredMesh}
${}$
\begin{lstlisting}
 class containing some meshing functionality
 complementary to Mesh_2d, Mesh_3d, Tree_2d and Tree_3d

\end{lstlisting}
\subsection{add\_element}
${}$
\begin{lstlisting}
 add an element with vertex indices, vertices already exist

\end{lstlisting}
\subsection{add\_vertex}
${}$
\begin{lstlisting}
 add a vertex

\end{lstlisting}
\subsection{angle}
${}$
\begin{lstlisting}
 interior angles of each element

\end{lstlisting}
\subsection{assign\_1d}
${}$
\begin{lstlisting}
 assign coordinatex (x0,y0) to containing element

\end{lstlisting}
\subsection{assign\_2d}
${}$
\begin{lstlisting}
 assign coordinatex (x0,y0) to containing element

\end{lstlisting}
\subsection{assign\_3d}
${}$
\begin{lstlisting}
 assign coordinatex (P0,y0) to containing element

\end{lstlisting}
\subsection{bnd\_1d}
${}$
\begin{lstlisting}
 left and right end points for 1D meshes

\end{lstlisting}
\subsection{boundary\_1d}
${}$
\begin{lstlisting}
 convert 1D mesh to 2D mesh

\end{lstlisting}
\subsection{boundary\_chain2}
${}$
\begin{lstlisting}
 get chained indices of boundary segments,
 used for setting up higher order polynomials along the boundary

\end{lstlisting}
\subsection{boundary\_length\_and\_direction}
${}$
\begin{lstlisting}
 edge length and direction of boundary segments
 TODO, this should be just edge length and direction

\end{lstlisting}
\subsection{cat}
${}$
\begin{lstlisting}
 concatenate two meshes

\end{lstlisting}
\subsection{chain\_1d}
${}$
\begin{lstlisting}
 chain 1D elements (segments)

\end{lstlisting}
\subsection{check\_dublicate\_elements}
${}$
\begin{lstlisting}
 check if elements are duplicate elements
 TODO, this does not check if elements cover each other, for example
 hierarchical meshes or ABC+BCD and ABD+ACD
 TODO check overlap by computation of area

\end{lstlisting}
\subsection{compute\_elem2elem}
${}$
\begin{lstlisting}
 set up element2element neighbourhood relation

\end{lstlisting}
\subsection{connect\_1d\_2d}
${}$
\begin{lstlisting}
 auto merge 1d and 2d mesh
 this silently requires that 1d segments consist at least of 3 elements
 TODO only implemented for triangles

\end{lstlisting}
\subsection{convert\_2d\_to\_1d}
${}$
\begin{lstlisting}

\end{lstlisting}
\subsection{copy}
${}$
\begin{lstlisting}
 copy constructor

\end{lstlisting}
\subsection{crop}
${}$
\begin{lstlisting}
 crop domain 

\end{lstlisting}
\subsection{cross\_section}
${}$
\begin{lstlisting}
 get cross-sections for 1D elements

\end{lstlisting}
\subsection{cut}
${}$
\begin{lstlisting}
 crop mesh to polygonal region

\end{lstlisting}
\subsection{delete\_element}
${}$
\begin{lstlisting}
 delete an element

\end{lstlisting}
\subsection{derivative\_matrix\_1d}
${}$
\begin{lstlisting}
 first order first derivative discretisation matrix on the 1d mesh

\end{lstlisting}
\subsection{derivative\_matrix\_2d}
${}$
\begin{lstlisting}
 first order first derivative discretisation matrix on the mesh

\end{lstlisting}
\subsection{derivative\_matrix\_2d\_2}
${}$
\begin{lstlisting}
 second order derivative matrix on a triangulation

\end{lstlisting}
\subsection{derivative\_matrix\_3d}
${}$
\begin{lstlisting}
 first order first derivative discretisation matrix on the mesh

\end{lstlisting}
\subsection{distance}
${}$
\begin{lstlisting}
 distance along edges from a point set to all other points

 open      : id of start point(s)
 countflag : if set use number of hops as distance not the euclidean distance

\end{lstlisting}
\subsection{dual\_mesh}
${}$
\begin{lstlisting}
 dual mesh formed by the centre of cicumference
 the dual mesh consists not only of triangles
 TODO rename in generate dual mesh

\end{lstlisting}
\subsection{edge\_length}
${}$
\begin{lstlisting}
 euclidean edge length

\end{lstlisting}
\subsection{edge\_midpoint}
${}$
\begin{lstlisting}
 edge mid-points

\end{lstlisting}
\subsection{edges\_from\_elements}
${}$
\begin{lstlisting}
 edges and boundaries from elements

\end{lstlisting}
\subsection{eigs}
${}$
\begin{lstlisting}
 eigenvalues of the lapalcian on the mesh

\end{lstlisting}
\subsection{elem2edge\_}
${}$
\begin{lstlisting}
 pointer of element to edge

\end{lstlisting}
\subsection{elem2elem\_matrix}
${}$
\begin{lstlisting}
 matrix with neighbourhood relations for each element

\end{lstlisting}
\subsection{element\_area}
${}$
\begin{lstlisting}
 area of elements
 1d elements have zero area and are not processed

\end{lstlisting}
\subsection{element\_centroid}
${}$
\begin{lstlisting}
 centroids of lements

\end{lstlisting}
\subsection{element\_midpoint}
${}$
\begin{lstlisting}
 barymetric centre of elements

\end{lstlisting}
\subsection{elements\_from\_edges}
${}$
\begin{lstlisting}
 2D elements from edges

\end{lstlisting}
\subsection{eval2pval}
${}$
\begin{lstlisting}
 element (centroid) value to vertex value
 TODO, use dual mesh or triangulation

\end{lstlisting}
\subsection{export\_delft3d\_net}
${}$
\begin{lstlisting}
 export into DFLOWFM delft3d net.nc file

\end{lstlisting}
\subsection{export\_msh}
${}$
\begin{lstlisting}
 export mesh in GMSH msh format

\end{lstlisting}
\subsection{export\_pos}
${}$
\begin{lstlisting}
 export triangles and vertex values to gmsh pos-file format (x,y,z,val)
 intended for re-meshing with values representing local mesh size

\end{lstlisting}
\subsection{export\_shp}
${}$
\begin{lstlisting}
 export edges to GIS shapefile
 each element as separate polygon with one z-value

\end{lstlisting}
\subsection{facing\_element}
${}$
\begin{lstlisting}
 get triangle ndx that is opposit, e.g. "facing" the vertex vdx of triangle tdx

\end{lstlisting}
\subsection{filter\_neighbour}
${}$
\begin{lstlisting}
 apply a function on the values on connected vertices

\end{lstlisting}
\subsection{find\_encroached\_edges}
${}$
\begin{lstlisting}
 find encroached edges in a triangulation,
 i.e. edges for which on of the two facing point false into their enclosing
 circle

\end{lstlisting}
\subsection{flip}
${}$
\begin{lstlisting}
 flip edges between two triangles
	flip
	for each side
		if (connection between opposit points shorter than between edges, swap edge)
		this-> flip
		that-> flip
	end

\end{lstlisting}
\subsection{flip\_global}
${}$
\begin{lstlisting}
 recursively flip edges, i.e ABC+BCD -> ABD+ADC,
 when new edge (diagonal) is shorter

\end{lstlisting}
\subsection{flip\_quality}
${}$
\begin{lstlisting}
 flip edges, when mesh quality constraint improves

\end{lstlisting}
\subsection{gaussmat\_2d}
${}$
\begin{lstlisting}
 matrix for gauss integration on a triangulation

\end{lstlisting}
\subsection{generate\_chews\_first}
${}$
\begin{lstlisting}
 triangulate domain with chew's first algorithm

\end{lstlisting}
\subsection{generate\_from\_centreline\_1d}
${}$
\begin{lstlisting}
 generate a mesh from centreline

\end{lstlisting}
\subsection{generate\_from\_centreline\_2d}
${}$
\begin{lstlisting}
 generate mesh from centreline
 TODO allow number of segments to change

 sets up a simple quadrilateral mesh in S-N coordinates
 centreline (must be sorted in streamwise direction)
 input variables:
 cS : S (streamwise) coordinates of centreline
 cL : N (spanwise) coordinate of left bank
 cR : N (spanwise) coordinate of right bank
 input variables controlling ouptut resolution:
 S  : S coordinate of slices in S-direction (diff(S) is element width)
        must be sorted in s-direction
 n  : n number of points per cross section
        (n-1) is number of elements per cross section
 output variables:
 mesh.{X,Y,S,N} : point coordinates
 mesh.T         : point indices of elements (corners of the quadrilaterals)
 -> make it orthogonal to banks by using a spline along n

\end{lstlisting}
\subsection{generate\_frontal}
${}$
\begin{lstlisting}

\end{lstlisting}
\subsection{generate\_ghost\_elements}
${}$
\begin{lstlisting}
 generate ghost elements, i.e. elements at the domain boundary, these 
 elements can overlap

 when the project flag set, ghost points are porjected to the boundary,
 the project flag is set for dual mesh generation
 the project flag is unset for application of the boundary condition

\end{lstlisting}
\subsection{generate\_gmsh}
${}$
\begin{lstlisting}
 generate a mesh from a polygon using gmsh

 inshp      : file name of shape file of preloaded shape file containing a polygon
 obase      : base of output file name
 resolution : struct containing default mesh resolution settings
 resfile_C  : file names of shape files, defining local resolution in polygonal regions
 opt        : options, see below

 this is a Static function

\end{lstlisting}
\subsection{generate\_hierarchical}
${}$
\begin{lstlisting}
 generate a hierachical mesh by recursively splitting elements
 containing boundary points

\end{lstlisting}
\subsection{generate\_triangle}
${}$
\begin{lstlisting}
 generate a mesh from a polygon using the programme "Triangle"

\end{lstlisting}
\subsection{generate\_uniform\_1d}
${}$
\begin{lstlisting}
 generate a uniformly spaced 1D mesh

\end{lstlisting}
\subsection{generate\_uniform\_quadrilateral}
${}$
\begin{lstlisting}
 generate a uniform 2D mesh

\end{lstlisting}
\subsection{generate\_uniform\_tetra}
${}$
\begin{lstlisting}
 uniformly tesselate a rhombic domain in 3D into tetrahedra

\end{lstlisting}
\subsection{generate\_uniform\_triangulation}
${}$
\begin{lstlisting}
 uniformly tesselate a rectangular (2d) domain into triangles

\end{lstlisting}
\subsection{get\_facing\_and\_shared\_vertices}
${}$
\begin{lstlisting}
 for a pairwise list (array) of triangles, determine there common and facing edges

\end{lstlisting}
\subsection{grid2tri}
${}$
\begin{lstlisting}
 topologically split a uniform mesh on a rectangular domain into triangles

\end{lstlisting}
\subsection{import\_delft3d\_net}
${}$
\begin{lstlisting}
 import mesh from Delft3d file ( {filanme}_net.nc )

\end{lstlisting}
\subsection{import\_msh}
${}$
\begin{lstlisting}
 import mesh from {filename}.msh files as generated by GSMH

\end{lstlisting}
\subsection{import\_triangle}
${}$
\begin{lstlisting}
 import a mesh generated with triangle (ele and node)

\end{lstlisting}
\subsection{improve\_iterative\_relocate\_insert}
${}$
\begin{lstlisting}
 iteratively improve the mesh by inserting vertices and smoothing
		fprintf('Iteration %d, %d elements, %d vertices, %d obtuse elements (%g%%)\n', iter, obj.nelem, obj.np, nobtuse, nobtuse./obj.nelem);

\end{lstlisting}
\subsection{improve\_iterative\_relocate\_uniform}
${}$
\begin{lstlisting}
 improve mesh by smoothing following by uniform refinement
		fprintf('Iteration %d, %d elements, %d vertices, %d obtuse elements (%g%%)\n', iter, obj.nelem, obj.np, nobtuse, nobtuse./obj.nelem);

\end{lstlisting}
\subsection{improve\_relocate\_global1}
${}$
\begin{lstlisting}
 iteratively improve angles to remove obtuse triangles

\end{lstlisting}
\subsection{improve\_relocate\_global2}
${}$
\begin{lstlisting}
 improve mesh globally

\end{lstlisting}
\subsection{improve\_relocate\_global\_3}
${}$
\begin{lstlisting}
 improve mesh quality globally

\end{lstlisting}
\subsection{improve\_relocate\_local}
${}$
\begin{lstlisting}
 iteratively improve angles to remove obtuse triangles

\end{lstlisting}
\subsection{improve\_relocate\_local\_old}
${}$
\begin{lstlisting}
 iteratively improve angles to remove obtuse triangles

\end{lstlisting}
\subsection{improve\_topology}
${}$
\begin{lstlisting}
 improve mesh topology

\end{lstlisting}
\subsection{insert\_mid\_points}
${}$
\begin{lstlisting}
 insert mid points into the mesh
 the new mesh is of much lower quality, but if all edges are flipped,
 this leads to the sqrt(2) refinement

\end{lstlisting}
\subsection{insert\_steiner\_points}
${}$
\begin{lstlisting}
 refine mesh by inserting steiner points (centre of circumference)
 for elements specified by tdx

\end{lstlisting}
\subsection{integrate\_1d}
${}$
\begin{lstlisting}
 integrate a quantity val across the mesh

\end{lstlisting}
\subsection{integrate\_discharge}
${}$
\begin{lstlisting}
 integrate discharge

\end{lstlisting}
\subsection{interp\_1d}
${}$
\begin{lstlisting}
 interpolate on a 1D mesh

\end{lstlisting}
\subsection{interp\_2d}
${}$
\begin{lstlisting}
 interpolate on a 2D mesh

\end{lstlisting}
\subsection{interp\_fourier}
${}$
\begin{lstlisting}
 interpolate values on the mesh using fourier methods

\end{lstlisting}
\subsection{interp\_tikhonov\_1d}
${}$
\begin{lstlisting}
 interpolation with Tikhonov regularisation

\end{lstlisting}
\subsection{interp\_tikhonov\_2d}
${}$
\begin{lstlisting}
 interpolation wiht Tikhonov regularisation in 2D

\end{lstlisting}
\subsection{interp\_tikhonov\_3d}
${}$
\begin{lstlisting}

\end{lstlisting}
\subsection{interpolate\_from\_boundary}
${}$
\begin{lstlisting}
 interpolate interior values from the boundary

\end{lstlisting}
\subsection{interpolate\_point}
${}$
\begin{lstlisting}
 interpolate from samples to mesh points by IDW method

\end{lstlisting}
\subsection{interpolation\_error\_1d}
${}$
\begin{lstlisting}
 estimate interpolation error in 1D

\end{lstlisting}
\subsection{interpolation\_error\_2d}
${}$
\begin{lstlisting}
 interpolate error in 2D

\end{lstlisting}
\subsection{interpolation\_error\_3d}
${}$
\begin{lstlisting}
 estimate interpolation error in 3D

\end{lstlisting}
\subsection{interpolation\_matrix\_1d}
${}$
\begin{lstlisting}
 linear interpolation matrix from mesh points to arbitrary coordinates P0

\end{lstlisting}
\subsection{interpolation\_matrix\_2d}
${}$
\begin{lstlisting}
 linear interpolation matrix from mesh points to arbitrary coordinates P0,y0

\end{lstlisting}
\subsection{interpolation\_matrix\_3d}
${}$
\begin{lstlisting}
 interpolation matrix for interpolation in 3D

\end{lstlisting}
\subsection{isacute}
${}$
\begin{lstlisting}
 determine acute triangles

\end{lstlisting}
\subsection{isobtuse}
${}$
\begin{lstlisting}
 determine obtuse triangles

\end{lstlisting}
\subsection{iterate\_smooth2}
${}$
\begin{lstlisting}
 iteratively improve the mesh by smoothing

\end{lstlisting}
\subsection{limit\_by\_distance}
${}$
\begin{lstlisting}
 max edge length
 minimum distance
 TODO, this will always be zero

\end{lstlisting}
\subsection{make\_elements\_ccw}
${}$
\begin{lstlisting}
 make all 2D elements clock wise (such that their area is positive)

\end{lstlisting}
\subsection{merge\_duplicate\_points}
${}$
\begin{lstlisting}
 merge duplicate points

\end{lstlisting}
\subsection{merge\_facing\_blunt\_triangles}
${}$
\begin{lstlisting}
 merge blunt triangles that face each other

\end{lstlisting}
\subsection{mesh1}
${}$
\begin{lstlisting}
 mesh in 1D

\end{lstlisting}
\subsection{mesh\_1d}
${}$
\begin{lstlisting}
 extract the 1d mesh

\end{lstlisting}
\subsection{mesh\_2d}
${}$
\begin{lstlisting}
 extract the 1d mesh

\end{lstlisting}
\subsection{mesh\_junctions}
${}$
\begin{lstlisting}
 mesh junctions of a channel network
		hold on

\end{lstlisting}
\subsection{n\_vertices\_1d}
${}$
\begin{lstlisting}

\end{lstlisting}
\subsection{nearest\_boundary}
${}$
\begin{lstlisting}
 determine nearest boundary segment for each input coordindate

\end{lstlisting}
\subsection{nedge\_}
${}$
\begin{lstlisting}

\end{lstlisting}
\subsection{nonobtuse\_refinement}
${}$
\begin{lstlisting}
 nonobtuse refinement according to Korotov
 not feasible for most obtuse triangles

\end{lstlisting}
\subsection{objective\_A}
${}$
\begin{lstlisting}
 one objective function value per angle

\end{lstlisting}
\subsection{objective\_T}
${}$
\begin{lstlisting}
 wrapper for mesh optimisation objective functions univariate in triangles

\end{lstlisting}
\subsection{objective\_angle}
${}$
\begin{lstlisting}
 objective function for iterative angle improvement

\end{lstlisting}
\subsection{optimum\_angle}
${}$
\begin{lstlisting}
 optimum angle for each vertex = 360^\deg / number of connected edges

\end{lstlisting}
\subsection{orthogonality\_quadrilaterals}
${}$
\begin{lstlisting}
 orthogonality condition for quadrilaterals

\end{lstlisting}
\subsection{path}
${}$
\begin{lstlisting}
 path along edges

\end{lstlisting}
\subsection{plot}
${}$
\begin{lstlisting}
 plot the mesh (and a discretised function) as a surface and net

\end{lstlisting}
\subsection{plot1d}
${}$
\begin{lstlisting}
 plot 1D mesh

\end{lstlisting}
\subsection{plot3}
${}$
\begin{lstlisting}
 plot mesh and values

\end{lstlisting}
\subsection{plotcs}
${}$
\begin{lstlisting}
 plot cross section

\end{lstlisting}
\subsection{project\_to\_boundary}
${}$
\begin{lstlisting}
 project a point to the boundary

\end{lstlisting}
\subsection{pval2eval}
${}$
\begin{lstlisting}
 vertex to element value

\end{lstlisting}
\subsection{quad2tri}
${}$
\begin{lstlisting}
 quadrilaterals to triangles

\end{lstlisting}
\subsection{raster\_boundary}
${}$
\begin{lstlisting}

\end{lstlisting}
\subsection{recover\_edges}
${}$
\begin{lstlisting}
 recover (boundary) edges

\end{lstlisting}
\subsection{refine}
${}$
\begin{lstlisting}
 refine by splitting marked triangles

\end{lstlisting}
\subsection{refine\_edge\_halving}
${}$
\begin{lstlisting}
 mesh refinement by longest edge bisection

\end{lstlisting}
\subsection{remove\_empty\_triangles}
${}$
\begin{lstlisting}
 remove degenerated triangles with zero area

\end{lstlisting}
\subsection{remove\_isolated\_vertices}
${}$
\begin{lstlisting}
 remove points that are not part of the mesh
 (gmsh leaves sometimes spurious points in the msh file)

\end{lstlisting}
\subsection{remove\_points}
${}$
\begin{lstlisting}
 remove points and associated elements

\end{lstlisting}
\subsection{remove\_quartered\_triangles}
${}$
\begin{lstlisting}
 point has connectivity 4 and is not on the boundary

\end{lstlisting}
\subsection{remove\_small\_islands}
${}$
\begin{lstlisting}
 delft3D requires islands to have at least 7 edges
 this functions splits edges surrounding small islands

\end{lstlisting}
\subsection{remove\_triply\_connected\_boundary\_vertices}
${}$
\begin{lstlisting}
 remove boundary vertices that are connected only to three vertices

\end{lstlisting}
\subsection{remove\_trisected\_triangles}
${}$
\begin{lstlisting}
 remove trisected trianges
 point has connectivity 3 and is not on the boundary

\end{lstlisting}
\subsection{renumber\_point\_indices}
${}$
\begin{lstlisting}
 renumber vertex indices

\end{lstlisting}
\subsection{resolve\_8\_vertices}
${}$
\begin{lstlisting}
 improve mesh by removing one edge from vertices with 8-edges
 (an interior vertex in a regular triangulation has 6 neighbours,
  and unstructured meshes with local refinement are possible with
  5 and 7 neighbours, 4,3, or 8 and more connected vertices are not necessary

\end{lstlisting}
\subsection{restore\_acuteness}
${}$
\begin{lstlisting}
 restore acuteness
 Laplacian smoothing may at some places decrease the mesh quality,
 this locally restores acute elements

\end{lstlisting}
\subsection{retriangulate}
${}$
\begin{lstlisting}
 retriangulate the mesh

\end{lstlisting}
\subsection{ruppert}
${}$
\begin{lstlisting}
 refine the mesh using ruppert's algorithm

\end{lstlisting}
\subsection{scale\_to\_boundary}
${}$
\begin{lstlisting}
 scale hierarchical mesh to match boundary coordinates
 experimental

\end{lstlisting}
\subsection{scatterplot}
${}$
\begin{lstlisting}
 scatterplot of data on mesh

\end{lstlisting}
\subsection{segment}
${}$
\begin{lstlisting}
 segment the mesh into parts according to laplacian eigenvalues

\end{lstlisting}
\subsection{smooth2}
${}$
\begin{lstlisting}
 Laplacian smoothing of vertex coordinates,
 replace every point by the average coordinate of its neibghbours

\end{lstlisting}
\subsection{smooth\_1d}
${}$
\begin{lstlisting}
 smoothes values in each reach
 does not smooth the values at the connection points

\end{lstlisting}
\subsection{smooth\_val}
${}$
\begin{lstlisting}
 smooth values on the mesh

\end{lstlisting}
\subsection{smoothness}
${}$
\begin{lstlisting}
 mesh smoothness as ratio of maximum edge length and minimum edge length

\end{lstlisting}
\subsection{split3}
${}$
\begin{lstlisting}
 split those triangles that contain a boundary point in three pieces,
 for hierrachical mesh generation

\end{lstlisting}
\subsection{split\_edge}
${}$
\begin{lstlisting}
 split an edge

\end{lstlisting}
\subsection{split\_edge\_perpendicular}
${}$
\begin{lstlisting}
 split edge perpendicularly

\end{lstlisting}
\subsection{split\_elem\_1d}
${}$
\begin{lstlisting}
 split a 1d element

\end{lstlisting}
\subsection{split\_encroached\_edges}
${}$
\begin{lstlisting}
 recursively split encroached edges

\end{lstlisting}
\subsection{split\_obtuse}
${}$
\begin{lstlisting}
 split obtuse elements

\end{lstlisting}
\subsection{split\_unsmooth\_edges}
${}$
\begin{lstlisting}
 split unsmooth edges

\end{lstlisting}
\subsection{statistics}
${}$
\begin{lstlisting}
 compute mesh statistics

\end{lstlisting}
\subsection{streamwise\_derivative\_matrix}
${}$
\begin{lstlisting}
 streamwise derivative matrix

\end{lstlisting}
\subsection{thalweg}
${}$
\begin{lstlisting}
 thalweg (deepest point along channel)

\end{lstlisting}
\subsection{to\_single}
${}$
\begin{lstlisting}
 TODO, also with indices

\end{lstlisting}
\subsection{uncross\_elements}
${}$
\begin{lstlisting}
 make sure, that 4 point elements span an area, and do not form a cross
 a call to this function should be succeeded by make_ccw
 this operator is idempotent

\end{lstlisting}
\subsection{uncross\_quadrilaterals}
${}$
\begin{lstlisting}
 make sure, that 4 point elements span an area, and do not form a cross
 a call to this function should be succeeded by make_ccw
 this operator is idempotent

\end{lstlisting}
\subsection{vertex\_distance}
${}$
\begin{lstlisting}
 connectivity of directly connected vertices

\end{lstlisting}
\subsection{vertex\_to\_edge}
${}$
\begin{lstlisting}
 connectivity matrix between vertices and adjacent edges

\end{lstlisting}
\subsection{vertex\_to\_element}
${}$
\begin{lstlisting}
 connectivity matrix between vertices and elements

\end{lstlisting}
\subsection{vertex\_to\_vertex}
${}$
\begin{lstlisting}
 connectivity matrix between vertices

\end{lstlisting}
\subsection{weighed\_laplacian\_smoothing}
${}$
\begin{lstlisting}
 weighed Laplacian smoothing

\end{lstlisting}
\subsection{xy2xys}
${}$
\begin{lstlisting}
 for boundary points: convert XY coordinate into a 1Dparametric coordinate,
 applied in mesh optimization, where movement of boundary points is
 constrained on the boundary

\end{lstlisting}
\subsection{xys2xy}
${}$
\begin{lstlisting}
 convert parametric 1D coordinate of boundary point back to cartesian XYc oordinate 

\end{lstlisting}
\section{grid/@Grid1}
\subsection{Grid1}
${}$
\begin{lstlisting}
 lump spatiotemporal data into a 1-dimensional grid

\end{lstlisting}
\subsection{binop}
${}$
\begin{lstlisting}
 operate function fun on data val within the context of a grid cell
 (for fitting grid cell values from sampled values)

\end{lstlisting}
\subsection{build\_index}
${}$
\begin{lstlisting}

 compute the grid-cell index for samples sampled at points X1

 name : name of the index field
 X1 : coordinate of source points
 R  : cut off radius (if not supplied ident to mesh width)

\end{lstlisting}
\subsection{fit}
${}$
\begin{lstlisting}
 lump (fit) sampled values into the corresponding grid cell

\end{lstlisting}
\subsection{predict}
${}$
\begin{lstlisting}
 interpolate from lumped data to specified location

\end{lstlisting}
\section{grid/@Grid2}
\subsection{Grid2}
${}$
\begin{lstlisting}
 lump spatiotemporal data into a 2-dimensional grid

\end{lstlisting}
\subsection{binop}
${}$
\begin{lstlisting}
 operate function fun on data val within the context of a grid cell
 (for fitting grid cell values from sampled values)

\end{lstlisting}
\subsection{build\_index}
${}$
\begin{lstlisting}

 compute the grid-cell index for samples sampled at points X1
 X1 : coordinate along first dimension
 X2 : coordinate along second dimension


\end{lstlisting}
\subsection{plot}
${}$
\begin{lstlisting}

\end{lstlisting}
\subsection{predict}
${}$
\begin{lstlisting}
 interpolate from lumped data to specified location

\end{lstlisting}
\section{grid/@Grid3}
\subsection{Grid3}
${}$
\begin{lstlisting}
 lump spatiotemporal data into a 3-dimensional grid

\end{lstlisting}
\subsection{build\_index}
${}$
\begin{lstlisting}
 compute the grid-cell index for samples sampled at points X1
 X1 : coordinate along first dimension
 X2 : coordinate along second dimension
 X3 : coordinate along third dimension

\end{lstlisting}
\section{mesh1d}
\subsection{dxspace}
${}$
\begin{lstlisting}

\end{lstlisting}
\subsection{dxspace2}
${}$
\begin{lstlisting}

\end{lstlisting}
\subsection{dzmesh}
${}$
\begin{lstlisting}

\end{lstlisting}
\subsection{mesh1}
${}$
\begin{lstlisting}

\end{lstlisting}
\subsection{mesh1d}
${}$
\begin{lstlisting}

\end{lstlisting}
\subsection{nlogstep}
${}$
\begin{lstlisting}

\end{lstlisting}
\section{optimization}
\subsection{improve\_smooth\_insert}
${}$
\begin{lstlisting}

\end{lstlisting}
\subsection{objective0\_angle1\_barycentric}
${}$
\begin{lstlisting}

\end{lstlisting}
\subsection{objective0\_angle2\_barycentric}
${}$
\begin{lstlisting}

\end{lstlisting}
\subsection{objective0\_angle2\_barycentric9}
${}$
\begin{lstlisting}

\end{lstlisting}
\subsection{objective0\_angle\_2\_cartesian}
${}$
\begin{lstlisting}

\end{lstlisting}
\subsection{objective0\_angle\_inf\_cartesian}
${}$
\begin{lstlisting}

\end{lstlisting}
\subsection{objective0\_barycentric9}
${}$
\begin{lstlisting}

\end{lstlisting}
\subsection{objective0\_pythagoras1\_barycentric9}
${}$
\begin{lstlisting}

\end{lstlisting}
\subsection{objective0\_pythagoras1\_cartesian}
${}$
\begin{lstlisting}

\end{lstlisting}
\subsection{objective0\_pythagoras2\_barycentric9}
${}$
\begin{lstlisting}

\end{lstlisting}
\subsection{objective0\_pythagoras2\_cartesian}
${}$
\begin{lstlisting}

\end{lstlisting}
\subsection{objective\_3\_angle}
${}$
\begin{lstlisting}

\end{lstlisting}
\subsection{objective\_A\_bnd}
${}$
\begin{lstlisting}

\end{lstlisting}
\subsection{objective\_P\_angle}
${}$
\begin{lstlisting}

\end{lstlisting}
\subsection{objective\_P\_angle\_scaled}
${}$
\begin{lstlisting}

\end{lstlisting}
\subsection{objective\_P\_angle\_scaled\_area}
${}$
\begin{lstlisting}

\end{lstlisting}
\subsection{objective\_P\_midpoint}
${}$
\begin{lstlisting}

\end{lstlisting}
\subsection{objective\_angle}
${}$
\begin{lstlisting}

\end{lstlisting}
\subsection{objective\_angle2\_barycentric}
${}$
\begin{lstlisting}

\end{lstlisting}
\subsection{objective\_angle\_p}
${}$
\begin{lstlisting}

\end{lstlisting}
\subsection{objective\_angle\_scaled\_area}
${}$
\begin{lstlisting}

\end{lstlisting}
\subsection{objective\_angle\_scaled\_circumference}
${}$
\begin{lstlisting}

\end{lstlisting}
\subsection{objective\_cosa}
${}$
\begin{lstlisting}

\end{lstlisting}
\subsection{objective\_cosa\_p}
${}$
\begin{lstlisting}

\end{lstlisting}
\subsection{objective\_cosa\_scaled\_side\_length}
${}$
\begin{lstlisting}

\end{lstlisting}
\subsection{objective\_distance\_edge\_centre}
${}$
\begin{lstlisting}

\end{lstlisting}
\subsection{objective\_distance\_edge\_centre\_perpendicular}
${}$
\begin{lstlisting}

\end{lstlisting}
\subsection{objective\_distance\_orthocentre\_excentre}
${}$
\begin{lstlisting}

\end{lstlisting}
\subsection{objective\_incentre\_excentre}
${}$
\begin{lstlisting}

\end{lstlisting}
\subsection{objective\_length\_min\_max}
${}$
\begin{lstlisting}

\end{lstlisting}
\subsection{objective\_length\_var}
${}$
\begin{lstlisting}

\end{lstlisting}
\subsection{objective\_thales}
${}$
\begin{lstlisting}

\end{lstlisting}
\subsection{objective\_thales\_difference}
${}$
\begin{lstlisting}

\end{lstlisting}
\subsection{test\_objective\_cosa\_p}
${}$
\begin{lstlisting}

\end{lstlisting}
\section{mesh}
\begin{lstlisting}
mesh generation, manipulation, analysis, refinement and optimization


\end{lstlisting}
\subsection{preload\_msh}
${}$
\begin{lstlisting}

\end{lstlisting}
\section{sparsemesh/@SparseMesh1}
\subsection{SparseMesh1}
${}$
\begin{lstlisting}
 lump time series of sampled spatial data in one dimension (projected)

\end{lstlisting}
\subsection{assign}
${}$
\begin{lstlisting}
 assign (lump) data "v0" sampled at sample times/location to field "field"


\end{lstlisting}
\subsection{assignS}
${}$
\begin{lstlisting}
 lump sequentially sampled data "v0" and assign to field "field"
 

\end{lstlisting}
\subsection{init}
${}$
\begin{lstlisting}
 initialize, segment sampling locations/times into blocks the sampled
 data is lumped to

\end{lstlisting}
\subsection{interp}
${}$
\begin{lstlisting}
 interpolate data stored in field "field" to coordinates Xi
 ingnore invalid data
 TODO, check if convex

\end{lstlisting}
\subsection{interpS}
${}$
\begin{lstlisting}
 interpolate data stored in field "field" to coordinates Xi,
 do not ignore invalid data

\end{lstlisting}
\subsection{rmse\_interp}
${}$
\begin{lstlisting}
 interpolation part of the error : 
 e ~ 1/2*d^2v/dx^2 * dx^2 + higher order terms
   ~ 1/2*d^2 v
 the other part of the error is the sampling error (gaussian noise)
 
 the mesh is optimal, when e_nois ~ e_interp


\end{lstlisting}
\section{sparsemesh/@SparseMesh2}
\subsection{SparseMesh2}
${}$
\begin{lstlisting}
 lump time series of sampled spatial data (track recordings) along two dimensions,
 e.g 1 projected spatial dimension and one for time time
 TODO : better blocks (all neighbours within mahalanobis distance)
 TODO : do not use simple mean, but allow for least squares regression
 TODO : precompute the least squares weights for accummarray

\end{lstlisting}
\subsection{assign}
${}$
\begin{lstlisting}
 assign (lump) data "v0" sampled at sample times/location to field "field"

\end{lstlisting}
\subsection{assignS}
${}$
\begin{lstlisting}
 lump sequentially sampled data "v0" and assign to field "field"

\end{lstlisting}
\subsection{init}
${}$
\begin{lstlisting}
 initialize, segment sampling locations/times into blocks the sampled
 data is lumped to

\end{lstlisting}
\subsection{interp}
${}$
\begin{lstlisting}
 interpolate data stored in field "field" to coordinates Xi
 ingnore data outside of the domain (convex interpolation)

\end{lstlisting}
\subsection{interpS}
${}$
\begin{lstlisting}
 interpolate data stored in field "field" to coordinates Xi,
 extrapolate beyond domain

\end{lstlisting}
\subsection{rmse\_interp}
${}$
\begin{lstlisting}
 interpolation part of the error : 
 e ~ 1/2*d^2v/dx^2 * dx^2 + higher order terms
   ~ 1/2*d^2 v
 the other part of the error is the sampling error (gaussian noise)
 
 the mesh is optimal, when e_nois ~ e_interp

 TODO this is e ~ f', not f''

\end{lstlisting}
\section{sparsemesh}
\begin{lstlisting}
lumping and interpolation of spatio-temporal data into a "mesh" that is spaced
optimally for the local density of sample points

allows for processing of large data sets with lower memory consumption and run time

intended for ADCP data processing

Overcomes the limitation of gridding, where some grid cells can have an insufficient
number of samples


\end{lstlisting}
\subsection{SparseMesh}
${}$
\begin{lstlisting}
 SparseMesh superclass

\end{lstlisting}
\section{test}
\subsection{test\_derivative\_matrices\_curvilinear}
${}$
\begin{lstlisting}

\end{lstlisting}
\section{mesh}
\begin{lstlisting}
mesh generation, manipulation, analysis, refinement and optimization


\end{lstlisting}
\subsection{trimesh\_fast}
${}$
\begin{lstlisting}

\end{lstlisting}
\end{document}
